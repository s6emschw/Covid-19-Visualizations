For our final EPP project, we produced a data set to track Covid-19 contact restriction policies at the federal level in Germany. Our efforts involved developing stringency indices that measured contact reduction in schools, private gatherings, and daily public activities. In conjunction with mobility data from Google’s Covid-19 Community Mobility Reports, we produced visualizations that reveal the effects of Germany’s Covid-19 lockdown policies. To produce a data set that accounts for covid-19 restrictions implemented at the state level would have required a great deal more research for which the timeline of our project did not permit. As such, the existing data set created for the purposes of our EPP final project only record Covid-19 policies suggested by the German federal government. For this reason, our data set and subsequent analysis do not take into consideration the lags during school and store re-openings among the states from the time the government announced the relaxation of restrictions until the states were able to actually start reopen certain aspects of public life. While some states were able to create social distancing plans that enabled schools, restaurants, and shops to reopen quickly after the first lockdown, others required more time to reopen. However, the initial data set we produced for our final project serves as a strong foundation from which the research team can begin to disaggregate the Covid-19 policy data in order to account for variations in the social distancing policy measures implemented at the state level during each phase of the pandemic. Doing so will thus produce a more fine-grained stringency policy index that better represents the degree of restrictiveness implemented at both the federal and state levels.
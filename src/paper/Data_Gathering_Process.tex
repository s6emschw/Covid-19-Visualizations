In order to obtain accurate information on any changes to social distancing policies announced by the German federal government, we searched through press releases concerning the developments of the Covid-19 outbreak published on the federal government website between February 2020 and February 2021. Given the dynamic nature of the government’s contact restrict measures, with policy adjustments announced every two to four weeks depending on the infection incidence level, we track the start and end date for each policy. While these official policy announcements carefully outlined regulations that restricted a wide range of daily public and private social encounters, our data set is limited to policy measures that influence social contact in schools, private gatherings, and public activities.

For each policy, we include a description of the measure, which is later used to create an ordinal scale of measurement that reports the restrictiveness of the policy. Also relevant is whether each policy is implemented federally or by an individual state. As time restrictions did not permit us to include state-level contact restriction policies, all entries in the existing data set track the suggested policy measures announced by the federal government. We additionally distinguish between policies that directly affect social contact in schools and those that impact other forms of contact including private gatherings and public activities. Finally, each entry includes links to the federal press release pages that announce the dates on which a given policy begins and ends.

In order to generate a representative account of the German school system, we produced sub-index score indicators that differentiate between four distinct grade levels including early childcare (E1), elementary school (E2), grades 5 to 10 (E3), and grades 11 to 12/13 (E4). We then repeated a similar procedure to observe aspects of public activities and private gatherings, respectively. To capture the effects of Covid-19 social distancing policies on public activities, we generated an additional four indicators to track restrictions on stores (S1), restaurants (S2), entertainment, sports and cultural facilities (S3), and finally religious gatherings (S4). Restrictions on private gatherings were summarized into a single indicator (R1), which observes the limitations on the number of individuals and/or households that may visit with each other at any given time.

After these preliminary efforts, we then used ordinal scales of measurement to rank stringency level for the individual policy phases of each sub-index score indicator. While the ordinal scales created for each indicator vary in terms of their maximum value, a value of zero always indicates no contact restriction; values of 1 or larger represent an increase in the policy’s level of contact stringency. For descriptions of the sub-index score indicators and their respective numerical scales, please view the codebook we provide as a separate document. The data set further assigns each policy entry three additional values, namely two binary variables, the flag and the recorded flag, as well as the maximum value of the policy’s affiliated indicator. While the flag variable distinguishes between indicators that are not characterized by their geographic scope (0) and those that are (1), the recorded flag indicates whether an indicator was implemented at the state (0) or national (1) level.